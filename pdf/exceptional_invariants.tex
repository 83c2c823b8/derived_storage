\RequirePackage{luatex85}
\documentclass[leqno]{ltjsarticle}
\usepackage{luatexja-fontspec}
\usepackage[top=10truemm,bottom=10truemm,left=20truemm,right=20truemm]{geometry}
\usepackage{luatexja} 
\usepackage{multicol,amsmath,amssymb,mathtools,ascmac,amsthm,amscd,physics,comment,dcolumn,titlesec,mathrsfs,mystyle,tikz-cd}
\usetikzlibrary{arrows.meta}
\titleformat*{\section}{\Large\bfseries}
\setlength{\parindent}{0pt}
\pagestyle{empty}
%\everymath{\displaystyle}
\begin{document}
We identify $\mathrm{SU}(2)$ with the group of unit quaternions
\[
\mathbb{H}_1=\{\,a+bi+cj+dk \mid a,b,c,d\in\mathbb{R},\ a^2+b^2+c^2+d^2=1\,\},
\]
via the homomorphism
\[
\rho:\ \mathbb{H}_1 \longrightarrow \mathrm{SU}(2)\subset \mathrm{SL}_2(\CC),\qquad
\rho(a+bi+cj+dk)=
\begin{pmatrix}
a+bi & c+di\\
-\,c+di & a-bi
\end{pmatrix}.
\]
In particular,
\[
\rho(i)=\begin{pmatrix} i&0\\0&-i\end{pmatrix},\quad
\rho(j)=\begin{pmatrix}0&1\\-1&0\end{pmatrix},\quad
\rho(k)=\begin{pmatrix}0&i\\ i&0\end{pmatrix}.
\]

\medskip

As unit quaternions,
\[
T^* = \Big\{\,\pm1,\ \pm i,\ \pm j,\ \pm k,\ \tfrac{1}{2}\big(\pm1\pm i\pm j\pm k\big)\,\Big\}.
\]
Matrix representatives are obtained via $\rho$.  
For example, generators can be chosen as
\[
\rho(i)=\begin{pmatrix} i&0\\[2pt]0&-i\end{pmatrix},\quad
\rho(j)=\begin{pmatrix}0&1\\[2pt]-1&0\end{pmatrix},\quad
\rho\!\left(\tfrac{1+i+j+k}{2}\right)=\tfrac{1}{2}
\begin{pmatrix}
1+i & 1+i\\[2pt]
-1+i & 1-i
\end{pmatrix}.
\]

\medskip

$O^*$ is generated by $T^*$ together with the additional unit quaternions
\[
\tfrac{1}{\sqrt{2}}(\pm1 \pm i),\quad
\tfrac{1}{\sqrt{2}}(\pm1 \pm j),\quad
\tfrac{1}{\sqrt{2}}(\pm1 \pm k),
\]
with all sign choices.  
Thus one generating set is
\[
\rho(i),\ \rho(j),\ \rho\!\left(\tfrac{1+i+j+k}{2}\right),\
\rho\!\left(\tfrac{1+i}{\sqrt{2}}\right),
\]
where
\[
\rho\!\left(\tfrac{1+i}{\sqrt{2}}\right)
=\tfrac{1}{\sqrt{2}}\begin{pmatrix}1+i&0\\[2pt]0&1-i\end{pmatrix}.
\]

\medskip

Let $\varphi=\tfrac{1+\sqrt{5}}{2}$ be the golden ratio and $\varphi'=\tfrac{1-\sqrt{5}}{2}=-\varphi^{-1}$.  
Then $I^*$ is generated by $Q_8=\{\pm1,\pm i,\pm j,\pm k\}$ together with the unit quaternions
\[
\tfrac{1}{2}\big(\pm \varphi \pm i \pm j \pm k\big),\qquad
\tfrac{1}{2}\big(\pm \varphi' \pm i \pm j \pm k\big),
\]
with sign choices constrained so that the norm equals $1$.  
A convenient generating pair is
\[
\rho(i)=\begin{pmatrix} i&0\\[2pt]0&-i\end{pmatrix},\quad
\rho\!\left(\tfrac{\varphi + i + j + k}{2}\right)
=\tfrac{1}{2}\begin{pmatrix}
\varphi + i & 1+i\\[2pt]
-1+i & \varphi - i
\end{pmatrix}.
\]

Let $T^*,O^*,I^* \subset \SL_2(\CC)$ be the binary tetrahedral, octahedral, and icosahedral groups.
For the natural action on $\CC[x,y]$, the invariant ring $\CC[x,y]^G$ ($G\in\{T^*,O^*,I^*\}$) is a hypersurface:
there exist homogeneous generators $X,Y,Z$ (of suitable positive degrees) such that, after a linear rescaling of variables over $\CC$, one has the weighted–homogeneous normal forms
\[
\begin{array}{ll}
\textbf{E$_6$ (for $G=T^*$):} & \displaystyle \CC[x,y]^{T^*}\ \simeq\ \CC[X,Y,Z]\big/\big(X^2 + Y^3 + Z^4\big), \\[0.6em]
\textbf{E$_7$ (for $G=O^*$):} & \displaystyle \CC[x,y]^{O^*}\ \simeq\ \CC[X,Y,Z]\big/\big(X^2 + Y^3 + YZ^3\big), \\[0.6em]
\textbf{E$_8$ (for $G=I^*$):} & \displaystyle \CC[x,y]^{I^*}\ \simeq\ \CC[X,Y,Z]\big/\big(X^2 + Y^3 + Z^5\big).
\end{array}
\]
These define the Kleinian (Du Val, simple) surface singularities of types $E_6,E_7,E_8$.

\medskip

\noindent\textit{Weights.}
Each equation is weighted–homogeneous; a convenient choice of weights is
\[
\begin{array}{lll}
E_6: & 2\,w_X=3\,w_Y=4\,w_Z=12 & \Rightarrow\ (w_X,w_Y,w_Z)=(6,4,3),\\[0.2em]
E_7: & 2\,w_X=3\,w_Y=w_Y+3\,w_Z=18 & \Rightarrow\ (w_X,w_Y,w_Z)=(9,6,4),\\[0.2em]
E_8: & 2\,w_X=3\,w_Y=5\,w_Z=30 & \Rightarrow\ (w_X,w_Y,w_Z)=(15,10,6).
\end{array}
\]

\medskip

\noindent\textit{Remark.}
Explicit Klein invariants $(X,Y,Z)$ for $T^*,O^*,I^*$ can be written down (classically via binary forms), but any such choice differs only by an invertible linear change and scalar rescaling; the resulting hypersurface is always one of the three normal forms displayed above.

\end{document}
