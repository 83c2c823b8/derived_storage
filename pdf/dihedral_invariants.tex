\RequirePackage{luatex85}
\documentclass[leqno]{ltjsarticle}
\usepackage{luatexja-fontspec}
\usepackage[top=10truemm,bottom=10truemm,left=20truemm,right=20truemm]{geometry}
\usepackage{luatexja} 
\usepackage{multicol,amsmath,amssymb,mathtools,ascmac,amsthm,amscd,physics,comment,dcolumn,titlesec,mathrsfs,mystyle,tikz-cd}
\usetikzlibrary{arrows.meta}
\titleformat*{\section}{\Large\bfseries}
\setlength{\parindent}{0pt}
\pagestyle{empty}
%\everymath{\displaystyle}
\begin{document}

Let $\zeta_n = e^{\frac{2\pi i}{n}}$.  
The binary dihedral group (often denoted $D_{n}$ in this context) is the subgroup of $\SL_2(\CC)$ generated by
\[
g=\begin{pmatrix}\zeta_n & 0 \\ 0 & \zeta_n^{-1}\end{pmatrix},
\qquad
h=\begin{pmatrix}0 & 1 \\ -1 & 0\end{pmatrix},
\]
satisfying the relations
\[
g^{n} = h^2 = -I, \qquad h g h^{-1} = g^{-1}.
\]
It has order $4n$, and is the preimage of the usual dihedral group of order $2n$ under the covering $\SL_2(\CC)\to\mathrm{PSL}_2(\CC)$.

\medskip

We use the same contragredient action of $\GL_2(\CC)$ on $\CC[x,y]$:
\[
(A\cdot f)(x,y) := f([x\ \ y]A^{-1}),\qquad A\in\GL_2(\CC).
\]

\medskip

\textbf{Step 1. Action of $g$.}  
As before,
\[
g\cdot x = \zeta_n^{-1}x,\qquad g\cdot y=\zeta_n y,
\]
so $x^i y^j$ transforms by $\zeta_n^{j-i}$.  
The $g$-invariants are generated by
\[
u:=x^n,\quad v:=y^n,\quad w:=xy,
\]
with relation
\[
uv-w^n=0,
\]
so
\[
\CC[x,y]^{\langle g\rangle} \simeq \CC[u,v,w]/(uv-w^n).
\]

\medskip

\textbf{Step 2. Action of $h$.}  
We compute $h^{-1}=-h$, so
\[
h\cdot (x,y)=(x,y)h^{-1}=(x,y)(-h)=(-y,x).
\]
Thus
\[
h\cdot x=-y,\qquad h\cdot y=x.
\]
On the invariants $u,v,w$ one finds
\[
h\cdot u = h\cdot (x^n)=(-y)^n = (-1)^n v,\qquad
h\cdot v = (x)^n = u,\qquad
h\cdot w = (-y)(x) = -xy = -w.
\]

\medskip

\textbf{Step 3. Dihedral invariants.}  
The full invariant ring $\CC[x,y]^{\langle g,h\rangle}$ consists of the $h$-invariant subring of $\CC[u,v,w]/(uv-w^n)$.  
From the action:
\[
h:\ (u,v,w)\mapsto((-1)^n v,\ u,\ -w).
\]
Therefore invariants can be taken as follows:

- If $n$ is even: $u+v$, $uv$, and $w^2$ are $h$-invariant, and the relation becomes $(uv)-(w^n)=0$.
- If $n$ is odd: then $u\mapsto -v$, $v\mapsto u$, $w\mapsto -w$, so invariants are generated by $u^2+v^2$, $uv$, and $w^2$, with relations inherited from $uv=w^n$.

\medskip

In particular, one obtains a presentation of the binary dihedral invariants:
\[
\CC[x,y]^{D_n}\ \cong\ 
\begin{cases}
\CC\big[u+v,\ uv,\ w^2\big]\ \big/\ \big((uv)-(w^n)\big), & n\ \text{even},\\[6pt]
\CC\big[u^2+v^2,\ uv,\ w^2\big]\ \big/\ \text{(relations)}, & n\ \text{odd}.
\end{cases}
\]

\medskip

Geometrically, $\Spec\big(\CC[x,y]^{D_n}\big)$ is the Kleinian surface singularity of type $D_{n+2}$.

\end{document}
