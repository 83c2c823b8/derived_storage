\RequirePackage{luatex85}
\documentclass[leqno]{article}
\usepackage{luatexja-fontspec}
\usepackage[top=10truemm,bottom=10truemm,left=20truemm,right=20truemm]{geometry}
\usepackage{luatexja} 
\usepackage{multicol,amsmath,amssymb,mathtools,ascmac,amsthm,amscd,physics,comment,dcolumn,titlesec,mathrsfs,tikz-cd,csvsimple}
\usetikzlibrary{arrows.meta}
\titleformat*{\section}{\Large\bfseries}
\setlength{\parindent}{0pt}
\pagestyle{empty}
%\everymath{\displaystyle}
%{\textbf{\Large{}}}\hspace{\fill} {\texttt{\Large{24A25030}}}{\Large{Yudai Yamamoto}}\\
\begin{document}

\section{English words}
\begin{table}[h]
  \centering
	\begin{tabular}{|p{5cm}|p{6cm}|p{5cm}|}
    \hline
		\textbf{English words} & \textbf{Meaning} &\textbf{Source}\\
    \hline\hline
    \csvreader[
      late after line=\\\hline, % add line after each row
      head to column names      % use first row as column names
    ]{englishwords.csv}{}% file name
		{\csvcoli  &\csvcolii &\csvcoliii}
  \end{tabular}
  \caption{English words}
\end{table}
\newpage

\section{Abbreviations}
\begin{table}[h]
  \centering
  \begin{tabular}{|p{5cm}|p{10cm}|}
    \hline
     \textbf{Abbreviation} & \textbf{Name} \\
    \hline\hline
    % Load rows from CSV
    \csvreader[
      late after line=\\\hline, % add line after each row
      head to column names      % use first row as column names
    ]{abbreviation.csv}{}% file name
    {\csvcoli  &\csvcolii}
  \end{tabular}
  \caption{Glossary terms from CSV}
\end{table}

\end{document}
